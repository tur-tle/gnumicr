\documentclass{article}
\usepackage{micr} %

% Force the actual MICR face for chunks (prevents any fallback)
\makeatletter
\newcommand{\MICR}[1]{%
  {\fontencoding{OT1}\fontfamily{GnuMICR}\selectfont \micr #1}%
}
\makeatother
% Per ISO 1004-1:2013: A=Transit, B=Amount, C=On-Us, D=Dash
\newcommand{\MICRTransit}[1]{\MICR{A#1A}}   % |: ..digits.. |:
\newcommand{\MICROnUs}[1]{\MICR{C#1C}}      % on-us ..digits.. on-us
\newcommand{\MICROnUsTerm}{\MICR{C}}        % single trailing on-us
\newcommand{\MICRAmount}{\MICR{B}}          % single amount symbol
\newcommand{\MICRDash}{\MICR{D}}            % internal separator
\begin{document}

Hello world this is a MICR test:
\\
MICR E-13B font of 14 characters. \\
1-0:\\
\micr{1234567890}\\
Per ISO 1004-1:2013\\
The 4 control characters in order are:\\
Transit, Amount,  On-us, and Dash.\\
A for Symbol 1; Transit \micr{A}\\
B for Symbol 2; Amount \micr{B}\\
C For Symbol 3; On-Us \micr{C}\\
D for Symbol 4; Dash \micr{D}\\

% Aux on-us (check #) — bracketed:
\micr{C}Check\micr{C}   \micr{A}Routing\micr{A}  Account\micr{C} \\

% Routing — bracketed by transit. % Account — digits + trailing on-us:
\MICROnUs{100123} \MICRTransit{123456789} \MICR{01234567890}\MICROnUsTerm

\end{document}

